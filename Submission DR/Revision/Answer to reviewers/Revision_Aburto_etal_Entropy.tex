\documentclass[a4paper,twoside, openright, 12pt, leqno]{article}
\usepackage[margin=0.5in]{geometry}

\title{Revision to the paper ``The threshold age of the lifetable entropy''}


\date{\today}
% Hint: \title{what ever}, \author{who care} and \date{when ever} could stand 
% before or after the \begin{document} command 
% BUT the \maketitle command MUST come AFTER the \begin{document} command! 
\begin{document}

\maketitle

\section*{Editor}
Both reviewers made a point of requesting more on why e0 and the threshold age are so similar. I would recommend that the authors respond to this comment. this response could be  (1) the addition of  some additional analysis explaining the connection, (2) the addition of  a footnote or a few sentences in the text explaining why the connection is not so clear and is a topic for future investigations, or (3) a response letter with the final manuscript explaining why you chose not to address this. I suspect that it is possible to show that in the case where mortality is Gompertz that the threshold very nearly approximates e0. The derivation probably can be done using the logic of Vaupel (1986), in which a/b is assumed to be negligible in size. Then, the empirical correspondence with e0 comes from the fact that mortality is roughly Gompertzian.
\linebreak

Reviewer 1’s point: What insight does the threshold age for entropy give above and beyond the threshold age for e dagger? Why would we calculate one over the other? also seems worth addressing in order to improve the motivation of the paper and attract more readers.


\section*{Reviewer A}
Main Strengths:
Overall, this is a clearly laid out and interesting paper. The mathematical relationships are clearly specified and easy to follow. \\

Main Weaknesses:
My only suggestions are for a bit more of a detailed discussion around the insight that the threshold age provides and around Figures 1 and 2. In particular:
\begin{itemize}
\item	What insight does the threshold age for entropy give above and beyond the threshold age for e dagger? Why would we calculate one over the other?
\item	Looking at Figure 1, the natural question is “do we always expect the threshold age to be so close to e0?” Why or why not?
\item	Figure 2 somewhat answers this question with a “no” given the trends in historical periods, but this is not really discussed in the text. Why the convergence of the entropy threshold age and other measures (e0 and a dagger) over time? What is driving this? Given modern mortality conditions, do we always expect the threshold age to continue to be very close to life expectancy? Or is it possible it will diverge again with improvements at older ages?
\end{itemize}




\section*{Reviewer B}
Main Strengths:
The paper presents interesting results related to an important measure for mortality studies, the Keyfitz entropy. The derivations are clear and provide useful and elegant results. These are certainly of interest to readers of the special collection on "Formal Relationships."\\

Main Weaknesses:
The point where the article could be improved is the "Applications" chapter. The authors could discuss in more details the interpretation of the threshold age for Keyfitz's, compared to e0 and the threshold age for e-dagger, as well as the implications of the findings of the paper for understanding mortality change and inequality of lifespans.\\

Figure 2 presents intriguing results, but are not clearly interpreted in the paper. A few examples of topics that would be interest to discuss are: what are the features of mortality curve that make e0 and the threshold age for Keyfitz's entropy to converge?; and why were them so different for early years?; why is the threshold age for Keyfitz's entropy smoother than the other measures, even in periods of great mortality variation?\\

Recommendations to Author:
In addition to the recommendation to expanding the interpretation of the results, there are a few minor suggestions:\\
On page 1, line 7, l(a) and l(x) are missing the t index.\\
On page 1, line 14, complement the sentence “remaining life expectancy at age a” with “at time t”.\\
On page 9, there is a typo in “Not the similarity”. \\
Review the last sentence of the conclusion, item (2). It doesn’t explain precisely what the threshold is about.


\end{document}
